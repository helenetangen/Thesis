\chapter{Background}\label{chapter:background}
This chapter contains the background for this thesis. In section \ref{section:wflo} the wind farm layout optimization problem is defined, and challenges of wind farm construction are presented. The first part of section \ref{section:ga} gives an introduction to genetic algorithms, this sub-section can be skipped if the reader is already familiar with genetic algorithms. The second part of section \ref{section:ga} introduces the 3 population distributed genetic algorithms that will be investigated in this thesis.


\section{Wind Farm Layout Optimization}\label{section:wflo}
This section aims to give the reader an understanding of the wind farm layout optimization problem. It is divided into two parts. Part one formally defines the wind farm layout optimization problem. Part two gives a brief introduction to wind farm construction with the goal of providing the reader with an understanding of its complexity and an introduction to the main challenges of wind farm layout design.


\subsection{The Wind Farm Layout Optimization Problem}
An overview of the wind farm layout optimization problem is presented by \cite{Samorani}. Grouping of wind turbines in a wind farm decreases installation- and maintenance costs. However, positioning of wind turbines in a farm also introduces new challenges. The power produced by wind turbines is largely dependent on wind speed, therefore it is important that the wind speed that hits each wind turbine is as large as possible. The main challenge for wind farms is that a wind turbine positioned in front of another will cause a wake of turbulence, meaning that the wind speed that hits the second wind turbine will be decreased. This effect is called \textit{wake effect}, and will be explained in the following sub-section. Since the goal is to produce as much power as possible, it is very important to position the wind turbines so that the wake effect is minimized. \cite{Samorani} stated the wind farm layout optimization problem like this ``The wind farm layout optimization problem consists of finding the turbine positioning (wind farm layout) that maximizes the expected power production". In this thesis, the problem formulation will be extended to include both cost constraints and optimization of the number of wind turbines, not just their positions. A formal definition is given below


\begin{quote}
\textit{''The wind farm layout optimization problem consists of finding the number of turbines and turbine positioning (wind farm) that maximizes the expected power production while minimizing costs.''}
\end{quote}


\noindent A typical solution to the wind farm layout optimization problem is given in figure \ref{figure:wind farm layout grady} \citep{Grady}. The rectangular grid represents the wind farm. Black cells represents positions containing wind turbines, and white cells represents empty positions. The wind farm layout in the figure is the optimal layout for a simple wind scenario where the wind is blowing at one direction with constant speed. When the wind is blowing from different directions with variable wind speed the turbines will be more scattered around in the grid.\\


\begin{figure}
\begin{center}
\includegraphics[scale=0.3]{images/"wind farm layout grady"}
\caption{Optimal layout for a simplified wind scenario \citep{Grady}. This figure is provided as an example of a wind farm layout.}
\label{figure:wind farm layout grady}
\end{center}
\end{figure}


\subsection{Challenges of wind farm construction}
\cite{Samorani} gives an overview of the main challenges of wind farm construction. First, a suitable site has to be found, meaning a site with good wind conditions. Wind power conditions are partitioned into 7 different wind power classes. With today's technology, sites with power class 4 or higher are considered suitable for hosting a wind farm. Even though the wind farm has the required wind conditions, it might not be suitable for hosting a wind farm because it might be far from the electronic grid, so that connecting it to it would be too costly, or it could require costly road work because current infrastructure is not able to handle the transportation trucks. Second, land owners has to be contacted and convinced that hosting a wind farm on their land is a good idea. Land owners usually gets a percentage of the wind farm profit. This phase of contract negotiation  usually takes a few months. At the same time, wind distribution need to be measured as accurately as possible. This step is extremely important, since the layout of the farm is optimized based on the measured wind distribution. Getting enough data to capture the wind distribution can take a few months if wind conditions are stable all year long, but if the wind conditions vary extensively over the year this step can take a few years. \\

\noindent An equally important step is to decide on which turbines to buy for the wind farm. A trade-off exists between power and cost since larger wind turbines are usually more expensive than smaller turbines, but they generally also produce more power. Realistic estimation of maintenance cost is also crucial in deciding on turbine type. In \cite{Samorani} the number of wind turbines are also decided in this step, but in this project, deciding the number of turbines is included in the wind farm layout optimization problem and will therefore be part of the next step. \\

\noindent Only after the site is found, wind turbine type is selected, and wind distribution is measured, can the layout optimization begin. Layout optimization faces different challenges. At some locations, the terrain is a great challenge because it contains areas which are unsuitable for wind turbines. These areas has to be dealt with by the layout optimization algorithm so that turbines are only positioned in legal areas. There are also constraints on how close turbines can be positioned, according to \cite{Sisbot}, the minimum spacing rule states that the minimum distance between turbines is 8D in prevailing wind direction, and 2D in cross wind direction, where D is the rotor diameter. Still, the greatest challenge of wind farm layout optimization is the wake effect. As mentioned above, the wake effect is the effect of reduced wind speed in the wake behind a wind turbine. Samorani explains the wake effect using the Jensen wake model \citep{Jensen}. As will be apparent after chapter \ref{chapter:relatedwork}, most research in wind farm layout optimization use the Jensen model because it is quite accurate and simple. The Jensen model will be explained below in order to give a brief understanding of how the wake effect is calculated. \\

\begin{figure}[h!]
\begin{center}
\includegraphics[scale=0.5]{images/"Wake Effect"}
\caption{The wake effect \citep{Samorani}.}
\label{Wake effect}
\end{center}
\end{figure}

\noindent In figure \ref{Wake effect} the small, black rectangle represents a wind turbine, and the blue area behind it illustrates the area that is affected by the turbulence created by the wind turbine. In the figure, the wind is blowing from left to right with uniform wind speed $U_0$. As the wind hits the wind turbine it creates a wake of turbulence behind it so that the wind speed at distance $x$ behind the wind turbine is $U < U_0$. The area behind the wind turbine that is affected by the wake at distance $x$ has the radius $r_1 = \alpha x + r_r$ where $r_r$ is the rotor radius and $\alpha$ is the entrainment constant, a constant that decides how fast the wake expands. For a detailed, mathematical explanation of the Jensen model and other wake models see references \citep{Jensen} and \citep{Liang}.\\

\noindent In summary, construction of a wind farm is a complicated and time consuming process. Consecutive important decision have to be made before one can even begin the layout optimization. The layout optimization is dependent on turbine cost, terrain parameters, wind conditions and turbine positioning. Finding the optimal layout is a non-linear, complex problem that only sophisticated algorithms can solve.


\section{Genetic Algorithms}\label{section:ga}
This section provides the background needed to understand genetic algorithms. The first sub-section gives a step-by-step introduction to simple genetic algorithms invented by \cite{Holland}. If not otherwise stated, this section is based on the books \cite{Holland} and \cite{Goldberg}. If the reader is familiar with genetic algorithms he or she can skip this sub-section. The second sub-section introduces the 3 population distributed genetic algorithms tested in this theses: The Island model, the Cellular model and the Pool model \citep{Gong}.\\


\subsection{Simple Genetic Algorithms (SGAs)}\label{subsection:sga}
Genetic algorithms are probabilistic search algorithms inspired by evolution. Figure \ref{GA} shows the five steps that constitute the genetic algorithm. The genetic algorithm operates on a population of individuals each representing a solution to the given problem. The first step is therefore to generate the initial child population, which usually consists of randomly generated individuals. The second step of the genetic algorithm is evaluation. The child population is evaluated based on some predefined fitness function (objective function); a measure of the goodness of the solution. Note that the terms fitness function and objective function will be used interchangeably in this thesis. The third- and forth step, adult selection and parent selection respectively, is the processes of deciding which individuals from the child population are allowed to grow up into adults and which are allowed to become parents for the next generation of individuals. These steps are naturally guided by the fitness of the individuals. The fifth and last step is called recombination. In this step, the genes of the parent population are recombined and altered in order to generate a new, hopefully improved, child population. After the new child population is generated the process starts again and continues until some predefined stop condition is reached. \\


\begin{figure}[h!]
\begin{center}
\includegraphics[scale=0.45]{images/"genetic algorithm"}
\caption{Overview of the five steps that constitute the genetic algorithm.}
\label{GA}
\end{center}
\end{figure}


\noindent Inspired by survival of the fittest, the population evolves into a population of better solutions to the given problem. Two properties are crucial for the utilization and success of the genetic algorithm: (1) One have to know how to measure the fitness of the individuals (goodness of the solutions), and (2) one have to find a way to represent individuals so that genetic operations can be performed on them. Examples of representation, fitness calculation, selection and genetic operations will be given below. Note that numerous different selection schemes and genetic operations exists and that the sub-sections below only provides one example of each in order to provide understanding of how a typical genetic algorithm works. How these steps are actually implemented in this thesis is explained in chapter \ref{chapter:methodology}. If the reader is familiar with simple genetic algorithms he or she can jump to sub-section \ref{subsection:dga}. \\ 


\paragraph*{Representation}
In genetics, an organism's hereditary information is called its genotype, and its observable properties its phenotype. For example, the hereditary information in your genes (genotype) are responsible for your eye color (phenotype). The genetic search algorithm usually works on genotypes represented as bit strings. \cite{Goldberg} explained this with a simple example. Let's say the objective function that we want to find the optimal solution for is $x^2$ for $x \in \{0, 31\}$. Then we can generate genotypes for the random solutions using bit strings of size 5, each representing a decimal value (phenotype) between 0 and 31. Figure \ref{Representation} displays the genotype and phenotype for four randomly generated individuals. Here, the phenotypes are just the genotypes on decimal form, but in other problems the phenotype could be everything from eye color to a wind farm layout.


\begin{figure}[h!]
\begin{center}
\includegraphics[scale=0.3]{images/"genotype-phenotype translation"}
\caption{Genotypes and phenotypes for 4 individuals where the phenotype is the decimal value of the genotype (binary number).}
\label{Representation}
\end{center}
\end{figure}


\paragraph*{Selection}
As mentioned above, selection is a two-step process: Adult selection and parent selection. Adult selection is the decision of choosing which children are allowed to grow up and enter the adult pool and thereby become potential parents for the next generation. The simplest form of adult selection is called \textit{full generational replacement} and it consists of simply replacing the previous adult pool with the new child population. This adult selection scheme is one out of three adult selection mechanism implemented for this thesis. These will all be explained in detail in chapter \ref{chapter:methodology}.\\

\noindent Parent selection is the process of selecting which individuals from the adult population that will be the parents of the next child population. The simplest form of parent selection is to simply select the fittest individuals. Unfortunately, this selection strategy often leads to premature convergence of non-optimal solutions. It is important to prioritize exploration, at least in the beginning of the search, otherwise, parts of the search space that could have lead to the optimal solution are cut off too soon. To cope with this problem \textit{controlled elitist selection} schemes are preferred. A very popular selection strategy is \textit{tournament selection} \citep{Razali}. In tournament selection, groups of \textit{n} individuals are randomly drawn from the population and the fittest individual in the group is appointed the tournament winner, and is therefore selected. Figure \ref{Tournament selection} illustrates how tournament selection works. In the example, \textit{n} is equal to 3, therefore the three individuals with fitness 9, 4 and 6 are randomly drawn from the population. Since in this thesis, the fittest individuals are those with lowest fitness value, the individual with fitness 4 wins the tournament and is chosen for reproduction.\\


\begin{figure}[h!]
\begin{center}
\includegraphics[scale=0.2]{images/"parent selection"/"tournament selection"}
\caption{Tournament selection. A group of three individuals are randomly drawn from the pool of all individuals. The best individual  in the group, the one with fitness 4, is selected for reproduction \citep{Razali}.}
\label{Tournament selection}
\end{center}
\end{figure}


\noindent By varying the value of \textit{n} you can control how much exploration your algorithm should do. If \textit{n} is equal to the population size only the best individuals are selected, and if \textit{n} is equal to 1 the search is completely random. This means that low values of \textit{n} leads to more exploration of the search space, something that can potentially guide the population to find better solutions. Higher values of \textit{n} on the other hand will often lead to fast convergence to sub-optimal solutions. These properties make it desirable to vary the value of \textit{n} during the genetic search so that exploration is prioritized at the beginning of the search, while exploitation (making use of seemingly good solutions) is prioritized at the end. Tournament selection is one out of two parent selection methods implemented in this thesis, these will also be explained in detail in chapter \ref{chapter:methodology}\\


\paragraph*{Crossover}
Crossover means combining genes of parent solutions to produce child solutions. One of the most common crossover schemes is called \textit{uniform crossover}. For each gene of the child solution there is a 50\% chance the gene will be inherited from the first parent and a 50\% chance that the gene will be inherited from the second parent. Figure \ref{Uniform Crossover} shows how uniform crossover works. As can be seen, the first gene of child 1 is inherited from parent 1, the second and third gene from parent 2, and the forth gene from parent 1 and so on. Uniform crossover is one out of three crossover methods implemented for this thesis.


\begin{figure}[h!]
\begin{center}
\includegraphics[scale=0.2]{images/crossover/"Uniform Crossover"}
\caption{Uniform crossover. Two child genotypes are created by combining the genotypes of two parents. Each gene is drawn from one of the parents with equal probability.}
\label{Uniform Crossover}
\end{center}
\end{figure}


\paragraph*{Mutation}
In biology, mutation is defined as a permanent alteration in the DNA sequence that makes up a gene. When the genetic algorithm works on genotypes of bit strings the process consists of simply flipping bits. Mutation is usually implemented by flipping some of the bits of the genotype as shown in figure \ref{Mutation}. \\


\begin{figure}[h!]
\begin{center}
\includegraphics[scale=0.2]{images/mutation/"mutation"}
\caption{Mutation of a single bit. The bit in position 6 at the upper bit string has the value 1 before the mutation, while after mutation the value is flipped to 0.}
\label{Mutation}
\end{center}
\end{figure}


\noindent Mutation is important because without mutation a population can evolve to a population of individuals where each genotype has the same value at a given position. Since every individual has the same value in their genotype at the same position, reproduction will never be able to make a new individual that does not also have the same value at the same position. With mutation however, there is always a probability of the value being flipped. Mutation is therefore crucial for maintaining diversity in the population and keeping it from becoming sterile.\\


\noindent Even though mutation is important, the probability of mutation needs to be kept low. If the mutation rate is very high, the genotype of a new individual will almost be a random bit string. Remember that a new individual is made by recombination of two individuals with good fitness in the previous population. If mutation changes the new individual heavily, it will not inherit the good features of its parents and the whole point of evolutionary search will be gone.\\


\subsection{Population Distributed Genetic Algorithms}\label{subsection:dga}
One of the main challenges of simple genetic algorithms is keeping diversity in the population long enough so that the population does not converge to a sub-optimal solution. By distributing the population, the algorithm is able to explore different solution paths. This is usually an advantage, because the solutions that obtains the best fitness in the end is not always the solutions that seem best right away. This section introduces the 3 different population distributed algorithms that are investigated in this thesis. For an in-depth introduction to the models see \citep{Gong}.\\


\paragraph*{The Island Model}
In the Island model, the population is divided into sub populations that are distributed onto different Islands. By letting each population evolve separately, different islands can explore different solutions. Figure \ref{Island model} displays a population divided into four sub-populations. \\ 


\begin{figure}[h!]
\begin{center}
\includegraphics[scale=0.3]{images/"population distributed genetic algorithms"/"island model"}
\caption{An Island model using a ring topology with four demes (Islands) of size five. \citep{Gong}}
\label{Island model}
\end{center}
\end{figure}


\noindent According to \cite{Huang}, six parameters must be specified when using the Island model. First of all, one need to decide on the number of demes (Islands). Second, the deme size needs to be specified; the number of individuals on each island. In figure \ref{Island model} the deme size is five, and four demes are used. Third, the topology must be specified; the allowed routes to migrate from one population to another. Numerous topologies can be used. In figure \ref{Island model} the arrows represents legal migration routes. Since the topology forms a circle it is called a ring topology. The forth and fifth parameters listed by Huang are migration rate and migration interval, meaning the number of individuals that migrate from one population to another and the number of generations between each migration respectively. These parameters are very important since they largely affect the time the population gets to explore different solutions before the best solutions from some of the demes takes over the population. Sixth, the policy of deciding which individuals that migrate, and how to replace existing individuals with new migrants needs to be specified. \\


\noindent The parameters listed above must be given careful thought when implementing the Island model, but as Gong explains, they are not the only ones. If the Island model is implemented in parallel one also have to decide if the migration is synchronous or asynchronous. Synchronous migration means that all migration is performed at the same time; at a specific generation. Asynchronous migration on the other hand, can be performed whenever one of the parallel processes are ready. Additionally, it has to be decided if the Island model is homogeneous or heterogeneous. By a homogeneous Island model, Gong et al. means an Island model where each sub population use the same selection strategy, genetic operations and fitness function, while as an heterogeneous Island model can implement different settings for different Islands.


\paragraph*{The Cellular Model}
Figure \ref{Cellular model} displays the Cellular model from \cite{Gong}. In the Cellular model the population is distributed in a grid of cells where each cell holds one individual. Each individual can only ``see'' the individuals of its neighborhood (as decided by the given neighborhood topology) and can only be compared to, and mate with individuals in its neighborhood. \\


\begin{figure}[h!]
\begin{center}
\includegraphics[scale=0.3]{images/"population distributed genetic algorithms"/"cellular model"}
\caption{Cellular model where the neighborhood topology consist of the cells to the left, right, over and under the given cell \citep{Gong}.}
\label{Cellular model}
\end{center}
\end{figure}


\noindent The takeover time is defined as the time it takes for one individual to propagate through the whole population. The neighborhood topology largely affects the takeover time. In figure \ref{Cellular model} the neighborhood topology is defined as only the individuals to the left, right, over and under the given individual. Since the topology includes a small number of individuals, the takeover time will be long, meaning that exploration is prioritized. If the topology consists of a larger number of cells the takeover time will, off course, be much shorter.\\


\noindent The Cellular model can also be implemented in parallel, ideally with one processor for each cell. As for the Island model, updating of the cells can be either synchronous or asynchronous.


\paragraph*{Pool Model}
Another population distributed model is called the Pool model. In this model the population is put in a shared global resource pool of $n$ individuals, where it can be accessed by different workers (threads). The resource pool is partitioned into equal size partitions, and each of these partitions are assigned to exactly one worker. Each worker perform selection, child generation and evaluation independently by selecting individuals from the entire resource pool, but only updating the positions in the resource pool for which it is responsible for. This process is demonstrated in figure \ref{Pool Model}. \\


\begin{figure}[h!]
\begin{center}
\includegraphics[scale=0.3]{images/"population distributed genetic algorithms"/"pool model"}
\caption{The pool model. Each worker has its own positions in the pool which it returns individuals to. The red processor is responsible for the red positions in the pool, and so on. Workers draw individuals from random positions in the pool, as indicated by the arrows, but can only write them back to its own positions, given that their fitness is higher than the fitness of the individual currently occupying the position \citep{Gong}.}
\label{Pool Model}
\end{center}
\end{figure}


\noindent A worker $w_1$ is responsible for $k$ positions in the pool. This is indicated by the coloring scheme in figure \ref{Pool Model}, where the worker and the positions it is responsible for has the same color. Each worker draws a population of individuals $i_1, i_2,...,i_k$ from random positions in the pool and performs genetic operations and fitness calculations on them. Next, each new child individual is written back to the worker's associated positions $1, 2,...,k$, given that its fitness is higher than the fitness of the individual currently occupying the position. The implementation details of this, and the other population distributed models, are presented in chapter \ref{chapter:methodology}.\\


\section{Structured Literature Review}\label{section:slr}
In order to get a deep understanding of genetic algorithms the books \cite{Holland} and \cite{Goldberg} was read. By reading the initial work of Holland, the inventor of the genetic algorithm, and the in depth explanation and analysis presented by Goldberg, the author got the proper understanding of genetic algorithms needed to write this thesis.\\

\noindent After an extensive literature search, the author was able to come up with the goal and research questions that will be investigated in this thesis. The literature search was performed in the following way: First, Google Scholar was used to search for articles with the search words \textit{genetic algorithms} and \textit{wind farm layout optimization}. A few of the first hits were read by the author. After reading these initial papers the author's understanding of the domain and domain vocabulary increased so new searches were made with the search words  \textit{genetic algorithm}, \textit{evolutionary algorithm}, \textit{genetic computation}, \textit{wind farm layout optimization} and a few more. The resulting papers were collected by the author and review based on the following metrics: (a) Number of citations, (b) similarity of the problem to the problem investigated in this thesis, (c) whether the article brings something new to the field, and (d) publishing date. Articles published in resent years were preferred, a long with a few initial papers because they laid the foundation for the field. The twelve articles summarized in table \ref{table:overview related work} were selected to give the reader an extensive understanding of wind farm layout optimization using genetic algorithms. Additionally, 4 articles about wind farm layout optimization using other artificial intelligence techniques were chosen to be presented in order to give the reader a brief overview over other approaches that has been used solving the given problem. \\

\noindent In summary, this chapter has introduced the wind farm layout optimization problem and the challenges that comes with wind farm construction. The simple genetic algorithm has been explained, and each of the 3 population distributed genetic algorithms have been introduced. The next chapter will give the reader an overview over the current research within wind farm layout optimization using genetic algorithms and a few other techniques.\\
