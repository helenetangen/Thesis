\chapter{Conclusion}\label{chapter:conclusion}


\noindent In this thesis, different genetic algorithm models have been implemented and tested. The goal of the thesis was to investigate how population distributed genetic algorithms would perform when used to solve the wind farm layout optimization problem. The performance of the population distributed genetic algorithm were compared to the performance of the simple genetic algorithm as well as to each other. In this chapter, an overview of the thesis will be presented in section \ref{section:overview}. In section \ref{section:goal evaluation} the goal and research questions are discussed. Section \ref{section:contributions} contains a discussion of how this thesis contribute to the field of wind farm layout optimization using genetic algorithm technology, and section \ref{section:further work} contains suggestions to further work.\\


\section{Overview}\label{section:overview}


\noindent In chapter \ref{chapter:background}, the wind farm layout optimization problem was introduced and challenges to wind farm construction were presented. A thorough review of the genetic algorithm was also given, and the 3 population distributed genetic algorithms investigated in this thesis were introduced.\\


\noindent Chapter \ref{chapter:relatedwork} contains a thorough review of research within the field of wind farm layout optimization using genetic algorithm technology. In table \ref{table:overview related work} the most important features of each publication is presented. Chapter \ref{chapter:relatedwork} also contain a brief overview of other artificial intelligence methods that have been used to optimize wind farm layout. During the research phase, it became apparent that no attempt had been made to investigate the performance of the Pool model and the Cellular model on wind farm layout optimization. The only population distributed genetic algorithm reported tested was the Island model \cite{Grady}, \cite{Huang}, \cite{Wan}, \cite{Sisbot} and \cite{Gao}. Although \citep{Grady} stated that by using the Island model they were able to come up with better solutions than when the simple genetic algorithm had been used, the models had not been compared when given the same amount of resources. These observations was the main motivation behind the choice to investigate the performance of different genetic algorithm models on the wind farm layout optimization problem.\\ 


\noindent Chapter \ref{chapter:methodology} presents the methodology for investigating the goal- and research questions. The decision was made to implement the genetic algorithm from scratch and connect it to the wind farm simulator provided by GECCO 2016. The chapter first gives a system overview. Next, the implementation details for the genetic algorithm are presented and discussed. The wind-, wake-, and power model, fitness function and test scenarios are also given. Additionally, the design decisions made when implementing the 3 population distributed genetic algorithms are presented and explained.\\


\noindent The final results of the experiments are presented in chapter \ref{chapter:results}. The first part of the chapter contains the results of testing different parameter values, selection mechanisms and recombination mechanism. The second part of the chapter contains the results obtained when running the different genetic algorithm models on 4 different wind scenarios, 2 containing obstacles and 2 not containing obstacles. A discussion of the results are provided at the end of the chapter. The results show that the Master/Slave model is able to find better results for each of the 4 scenarios. The Pool model performs second best, the Island model third best and the Cellular model forth best.\\


\section{Goal Evaluation}\label{section:goal evaluation}


\noindent The goal of this thesis was to investigate the possible advantages of using population distributed genetic algorithms in optimizing wind farm layout. In this section, the research questions will be reviewed, answered and discussed.\\


\noindent \textbf{Research question 1: } \textit{Can distributed genetic algorithms improve the quality of the solution to the wind farm layout optimization problem as compared to the simple genetic algorithm?}\\


\noindent The results from this thesis show that population distributed genetic algorithm does not necessarily find better solutions than the simple genetic algorithm for the wind farm layout optimization problem. The Master/Slave model consistently found better wind farm layouts than the population distributed genetic algorithms. \\


\noindent As mentioned in chapter \ref{chapter:introduction} the wind farm layout optimization problem is impossible to solve analytically because the search space is too big. For a wind farm site with 500 legal turbine positions 2$^{500}$ different layouts exist, that is, more layouts than the number of particles in the universe! This analogy explains just how complex the wind farm layout optimization problem is, and it illustrates how unlikely it is that a genetic algorithm is able to come up with the global optimal solution. The Master/Slave model is the model that spends less time exploring different number of turbines. It settles on a given number of turbines after approximately 10 generations and starts optimizing the turbine positions within that solution space. This strategy of settling for a local minima fast and optimize turbine positions for as many generations as possible seem to work! And, as the results imply, this might be the property that makes the Master/Slave  model outperform the other models. This explanation is reinforced by the results obtained in section \ref{section:parameter settings}. It was shown that by decreasing the time it takes for the best solutions to monopolize the population, the algorithm was able to come up with better solutions.\\


\noindent Another possible explanation why the Master/Slave model obtains better results than the population distributed genetic algorithms could be that the genetic algorithm parameters and selection strategies was optimized running this model. It might be the case that the population distributed models could obtain better results if these settings where tailored for every model. Sadly, the time frame of this thesis prevented the author to perform these experiments for every model. \\


\noindent It is important to note that this thesis have only proven that the Master/Slave model performs better than population distributed genetic algorithm when each model is provided the same amount of resources and with the given settings and design decisions used in this thesis. These results should not be interpreted as absolute, but that under the given circumstances the Master/Slave model is able to get better results. It could be argued that it is wrong to compare the population distributed genetic algorithms with the simple genetic algorithm with the same amount of resources. The essential property of population distributed genetic algorithms is that they spend more time exploring different number of turbines before they settle down on a given number and start optimizing turbine positions. It would be interesting to observe if letting the algorithms run for the same number of generations after they have settled on a given number of turbines would improve the results of the population distributed genetic algorithms. However, this seems unlikely for every model except from the Island model. As the results showed, the Island model was the only model that was able to explore different number of turbines for a significant amount of time, the other population distributed models settled almost as fast as the Master/Slave model. A problem with the suggested strategy is that it is very difficult to compare models without providing them with the same amount of resources.\\


\noindent Another fact that might explain why the population distributed genetic algorithms were not able to come up with results as good as the Master/Slave model could be that the parameters that are specific for a given model was not optimized. For the Pool model, the number of workers were not optimized, the number of Island, Island topology and migration strategy could be optimized for the Island model and the topology should also be optimized for the Cellular model. In this thesis these values are set based on the authors previous experience with genetic algorithms because there was not time to optimize these settings for each model.\\


\noindent \textbf{Research question 2: } \textit{Which distributed genetic algorithm works best for the wind farm layout optimization problem? What properties are essential for its success?}\\


\noindent Out of the population distributed genetic algorithms, the Pool model was able to find the best solutions. The Island model also came up with acceptable results, while the Cellular model had trouble keeping up with the other models.\\


\noindent Above it was discussed that the property that made the Master/Slave model superior to the other models could be that it does not waste time searching for the optimal number of turbines, but settles fast in a local minima, and spends most of its time optimizing wind turbine positions. The results obtained by the Pool model supports this explanation. The Pool model is the population distributed genetic algorithm that resembles the Master/Slave model most. The results show that it settles on a given number of turbines almost as fast as the Master/Slave model and spend an extensive amount of time optimizing turbines positions. The difference in fitness between the Master/Slave model and the Pool model is small enough so that it might seem reasonable to conclude that if the settings had been tailored for the Pool model, it might would be able to obtain similar results.\\


\noindent As mentioned, the Island model was the only model able to take advantage of its population distributed property by using it to explore different number of turbines. It was expected that this property would be crucial for the success of the populating distributed genetic algorithms. However, this thesis have proven that under the given circumstances this property was not an advantage. As discussed before a number of reasons could explain this result. First, it might be the case that this property is only advantageous when the model is allowed to run for more generations, and is allowed to optimize each solution longer. It might also be the case that this exploration is a waste of time because the algorithms are not able to find the global optimal solution, and should concentrate on optimizing a local minima.\\


\noindent The Cellular model was not able to come up with good solutions for the problem. The model performed so much worse than the other models that it is safe to say that it is not suited for wind farm layout optimization. It is structured in a way that requires way more time for good solutions to spread in the population that it would be too time consuming to use this model in wind farm optimization. The computational load is already extremely high, so increasing the number of generations so extensively that the Cellular model could stand a chance against the other models is too time consuming, at least on a personal computer. \\


\noindent In summary, genetic algorithms have been used trying to solve the wind farm layout optimization problem for more than 20 years. Genetic search algorithms have become the primary artificial intelligence technique in wind farm optimization because the problem is too complex to solve analytically. However, the complexity of the problem is also apparent when genetic algorithms are used. It is shown that properties that usually are advantageous for genetic algorithms such as the property of slowing down convergence in order to explore different solutions might not benefit in wind farm optimization. However, the suggestions presented in \ref{section:further work} should be investigated before firm conclusions can be made.\\


\section{Contributions}\label{section:contributions}


The research presented in chapter \ref{chapter:relatedwork} have shown that genetic algorithms are suitable for wind farm layout optimization. Improvements such as improvements to the genetic algorithm, the fitness function, and the power model, have been investigated. \citep{Grady} showed that using the Island model on the scenarios provided by \citep{Mosetti} lead to better results. However, no attempt, as far as the author knows, have been done in investigating the performance of the Pool model and Cellular model on the wind farm layout optimization problem. Therefore, this thesis contributes to the field by being the first project attempting to show which, if any, of the population distributed genetic algorithms are best equipped for wind farm layout optimization. The results obtained in this thesis can be used as a basis for further research within the field. Section \ref{section:further work} contains suggestions on how the research presented in this thesis could be investigated further. Many interesting directions can, and should, be taken in order to find out which method is truly the primary method for wind farm layout optimization.\\


\noindent The main lesson from this project:\\


\begin{quote}
\textit{Using population distributed genetic algorithms to solve the wind farm layout optimization problem will not necessarily lead to better fitness than with simple genetic algorithms. Population distributed genetic algorithm demand more computational resources and will therefore have trouble beating a simple genetic algorithm with the same amount of resources. Population distributed genetic algorithms are more advanced than simple genetic algorithms and it will therefore be more difficult to find the correct settings for them. However, when handed more resources it is clear that population distributed genetic algorithms are powerful algorithms that should be investigated further on the wind farm layout optimization problem.}
\end{quote}


\section{Further work}\label{section:further work}
As far as the author knows, this is the first time different population distributed genetic algorithms have been compared against each other and against a simple genetic algorithm in their ability to solve the wind farm layout optimization problem. This thesis can be view as a first attempt to understand how population distributed genetic algorithm can affect wind farm layout optimization, and it should be explored further.\\


\noindent Both research questions should be further investigated. As has been mentioned, further work should be done in optimizing the parameters that are specific for the each of the population distributed models. For the Island model experiments should be performed to optimize \textit{deme size}, \textit{deme count}, \textit{migration rate}, \textit{migration interval} and the \textit{topology}. Especially, the author believes that optimizing the \textit{migration interval} would have a major impact on the performance of the Island model. As the results showed, the Island model was the only model able to explore different number of turbines for a significant amount of time. However, it was not given enough time to optimize each of these number of turbines, and therefore it was not able to realize the potential of the solution it has found before it jumped to another solution. Increasing the migration interval might have changed the final results. For the Cellular model, different topologies should be investigated and compared. A larger square, or another form of topology would affect the results. The replacement strategy of replacing every individual at every generation even thought the new individual had worse fitness than the previous individual should probably be revised. It might also be the case that a different parent selection function, such as roulette wheel, would work better for this model, even though it did not work for the Master/Slave model. For the Pool model, different number of workers should be tested. It would also be very interesting to take control over the scheduling of the different workers, and investigate the effect of letting one or more of the workers to get far ahead of the others in evolution and compare this to simulations where the threads are at about the same generation. \\


\noindent In addition to optimize the parameters that are specific for each model it would also be desirable to redo the parameter settings from section \ref{section:parameter settings} for each of the population distributed genetic algorithms. This would make sure that the Master/Slave model does not have an advantage because the settings used for every model is only tested on that model. Even though the author does not believe this part to be as important as the improvements  recommended above, it could have an impact on the final results. \\


\noindent All 3 population distributed genetic algorithms were implemented as homogeneous models. It would be very interesting to observe how implementing these as heterogeneous models would affect the results. This would require an extensive amount of work because so many different settings could be used, that it might be wise to start with only one of the models. It would be very interesting to observe how the Island model could be improved by using different parameter settings, selection mechanisms and recombination methods on each Island and compare the results against a homogeneous implementation. It would also be very interesting to let the workers of the Pool model run with different settings, and to let different settings apply to different parts of the grid in the Cellular model.\\


\noindent The Island model and Cellular model were both synchronous. Implementing these as asynchronous models could also be interesting to investigate, although the author does not believe that this would impact the results as much as implementing heterogeneous models.\\


\noindent Since this thesis is a contribution to GECCO 2016, where the goal is to optimize the objective function provided by the contest a multi-objective function was not implemented. However, it would be very interesting to solve the wind farm layout optimization problem with a multi-objective genetic algorithm, and combine this with population distributed genetic algorithms. Finding the best objective function is a difficult job, which includes a lot of estimation. It is extremely difficult to come up with a fitness function that guide the population to the optimal solution. Because of this, it is often more accurate to use a multi-objective function because then you do not have to weight the importance of the different objective against each other. A population distributed genetic algorithm would be perfect for optimizing a mulit-objective function because each Island, worker or area could be assigned a different objective function.\\


\noindent \citep{Gong} also present population distributed genetic algorithms which are combinations of each other. The Island model implemented for this thesis was actually one of the models mentioned, because the Master/Slave model was run on each Island. However, the population distributed genetic algorithms were not combined. \citep{Gong} suggest combining the Island model with the Cellular model (one cellular model on every Island) or running an Island model on each Island in the Island model. These models go even further in their attempt to slow down take over time, and investigate different solutions.\\


\noindent \citep{Saavedra-Morena} used a greedy heuristic to decide the initial position of the wind turbines, and then used the genetic algorithm to improve the results. \citep{Saavedra-Morena} showed that the final results were better than those obtained by the genetic algorithm when it started out with random solutions. This strategy could also be implemented together with the models implemented in this thesis to investigate if this would improve the final results obtain by the population distributed genetic algorithms. \\


\noindent In section \ref{section:relatedworkother} it was shown that greedy heuristics, simulated annealing, ant colony algorithms and other swarm algorithms could also be used to solve the wind farm layout optimization problem. A study comparing different artificial intelligence methods in solving the wind farm layout optimization problem would be very interesting. It would also be possible to combine the different algorithms either by letting a genetic algorithm run first at trying to let one of the other methods improve the final results, or the other way around. \\

\noindent As discussed, this thesis has answered some questions and left the author with many new questions that it would be very interesting to investigated and answer. This thesis has laid the foundation for further work within population distributed genetic algorithms and wind farm layout optimization.\\