\section*{Sammendrag}

\noindent \textit{(This is a Norwegian translation of the abstract.)}\\

\noindent Vindmølleteknologi er en lovende kilde til fornybar energi. Likevel kan ikke potensialet til vindmøllene utnyttes hvis ikke posisjoneringen av vindmøllene i vindmølleparken er optimalisert. Utfordringen med vindmølleposisjonering er at løsningrommet - antall mulige vindmølleposisjoneringer - er enormt, og det er ekstremt vanskelig, om ikke umulig, å finne den optimale posisjoneringen analytisk. Genetiske algoritmer har vist seg å være svært nyttige når de har blitt brukt til å løse ikke-lineære optimaliseringproblemer, og de har vist lovende resultater brukt i vindmølleposisjonering. Det er likevel lite forskning på hvorvidt populasjonsdistribuerte genetiske algoritmer kan oppnå bedre resultater enn ikke-populasjonsdistribuerte algoritmer for vindmølleposisjonering. \\

\noindent I dette prosjektet har det blitt undersøkt om populasjonsdistribuerte genetiske algoritmer kan løse vindmølleposisjoneringsproblemet bedre enn den tradisjonelle genetiske algoritmen. Prosjektet er et bidrag til en konkurranse for vindmølleposisjonering lansert av konferansen GECCO (the Genetic and Evolutionary Computation Conference) 2016.\\
    
\noindent En avansert genetisk algoritme bestående av fire forskjellige modeller var implementert for dette prosjektet. En av de genetiske algoritmene var en tradisjonell genetisk algoritme, mens de tre andre var populasjonsdistribuerte genetiske algoritmer. Programmer ble koblet sammen med GECCO 2016's vindfarmsimulator slik at de forskjellige genetiske algoritmene kunne bli observert og sammenlignet på forskjellige vindsenarioer.\\
    
\noindent Resultatene viste at den tradisjonelle genetiske algoritmer konsekvent fant bedre vindmølleposisjoneringer enn de tre populasjonsdistribuerte modellene. To av de populasjonsdistribuerte algoritmene gav lovende resultater, mens den tredje viste seg å være uegnet til å finne tilfredstillende vindmølleposisjoner. Resultatene ledet til konklusjonen at populasjonsdistribuerte algoritmer ikke alltid klarer å levere like gode resultater som den tradisjonelle genetiske algoritmen når hver modell er tildelt de samme resursene.\\

\newpage