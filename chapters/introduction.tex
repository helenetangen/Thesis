\chapter{Introduction}\label{chapter:introduction}
This thesis is a contribution to the third edition of the wind farm layout optimization contest launched by the Genetic and Evolutionary Computation Conference (GECCO) 2016. The contest involves optimizing the number of turbines and turbine positioning in a wind farm with goal of producing maximum power to minimum cost using genetic algorithms. The wind farm simulator is provided by GECCO 2016, hence, the focus of this thesis will be on improving the genetic algorithm that will be used to search for the optimal wind farm layout. This is accomplished by investigating the state of the art within wind farm layout optimization and genetic algorithms and experiment with variations of the genetic algorithm.\\

\noindent This chapter provides an introduction to this thesis. Section \ref{section:motivation&background} contains the background and motivation. Section \ref{section:goal&researchquestions} introduces the goal and research questions that will be investigated. In section \ref{section:researchmethod} a brief overview of the research method is given. Contributions is given in section \ref{section:introduction contributions}, and an overview of the thesis structure is given in section \ref{thesisstructure}.


\section{Motivation and Background}\label{section:motivation&background}
Transitioning from non-renewable energy sources to renewable energy sources is one of the largest, if not the larges political challenge of today. Renewable energy is less polluting than non-renewable energy and should therefore be preferred. However, renewable energy sources make up only 21\% of the world's energy sources as of 2011 \citep{url1}. Wind turbine technology is a promising source of renewable energy. Advances in wind turbine technology have led to increased power production at lower cost. Despite these improvements, wind turbines still produce less energy than predicted because of the wake effect; reduction in wind speed caused by turbines placed in front of other turbines \citep{Samorani}. For wind energy to become a bigger player in the world's energy sources, sophisticated methods for wind turbine placement in wind farms need to be developed so that each turbine produces as much energy as possible. \\

\noindent Wind turbine positioning is hard to optimize analytically. Fortunately, a wide variety of local search methods and bio-inspired methods have shown promising results, with genetic algorithms being the most popular method. As more advanced approaches to evaluate layouts has been developed, and more realistic constraints are introduced, more sophisticated genetic algorithms are required. To come up with more advanced genetic algorithms for solving the wind farm layout optimization problem, the annual Genetic and Evolutionary Computation Conference (GECCO) 2016, launched a competition where different contestants will provide their own implementation of a genetic algorithm \citep{url2}. The goal of the competition is to bring more realistic problems to algorithm developers, and to create an open source library useful beyond the scope of the competition. Wind parameters and evaluation mechanisms are provided by GECCO 2016, therefore, the focus of this thesis will be on optimizing the genetic algorithm. Still, some knowledge of wind turbines, wind farms, and wind- and power models are crucial to understanding this thesis and will therefore be introduced in chapter \ref{chapter:background}. \\

\noindent Greedy heuristics, simulated annealing search, ant colony algorithms, particle swarm optimization and genetic algorithms have all been used in solving the wind farm layout optimization problem, and these will all be reviewed in chapter \ref{chapter:relatedwork}. Turbine positioning has been improved by genetic algorithms for more than 20 years, each approach bringing something new to the field such as a new type of genetic algorithm, a more realistic environment, or combinations of genetic algorithms and other approaches. A few researchers have solved the wind farm layout optimization problem by implementing a population distributed genetic algorithms called the Island model, and it has shown promising results. However, as far as the author knows, no attempt has been made in implementing any of the other existing population distributed models. This is the main motivation behind this thesis, and has inspired the author to investigate the effect different population distributed algorithms can have on the wind farm layout optimization problem. Chapter \ref{chapter:background} contains a review of the population distributed genetic algorithm investigated in this thesis.


\section{Goal and Research Questions}\label{section:goal&researchquestions}
This section states the goal statement and research questions that will be investigated in this thesis. \\

\noindent \textbf{Goal statement}

\begin{quote}
\textit{The project goal is to investigate the potential advantages of using population distributed genetic algorithms in optimizing wind farm layout, i.e. solving the wind farm layout optimization problem.} \citep{Samorani}
\end{quote}

\noindent The performance of population distributed genetic algorithms will be observed and compared to the performance of a simple genetic algorithm (non population distributed genetic algorithm) in order to answer the first research question. In addition, each of the population distributed genetic algorithms will be compared against the others with the goal of answering the second research question. \\

\noindent \textbf{Research question 1}

\begin{quote}
\textit{Can distributed genetic algorithms improve the quality of the solution to the wind farm layout optimization problem as compared to a simple genetic algorithm?}
\end{quote}

\noindent \textbf{Research question 2}

\begin{quote}
\textit{Which distributed genetic algorithm works best for the wind farm layout optimization problem? What properties are essential for its success?}
\end{quote}

\noindent Both research questions will be answered by testing the different distributed genetic algorithms in a wind farm simulator provided by GECCO 2016. The genetic algorithms will be implemented from scratch in order to give the author control of the environment, as well as the possibility of adding functionality fast and easy. Each genetic algorithm will be tested on different wind scenarios which are also provided by the contest. Research question 1 will be answered by implementing 3 different types of population distributed genetic algorithms and compare their results with a simple genetic algorithm on different wind scenarios. Research question 2 will be tested the same way, and the results from each of the population distributed algorithms will be compared.


\section{Research Method}\label{section:researchmethod}
In order to answer the research questions, a genetic algorithm was implemented for this thesis. Even though a simple genetic algorithm were provided by GECCO 2016 along with a wind farm evaluator and wind scenarios, the choice was made to implement the algorithm from scratch so that the population distributed genetic algorithms could be added easily, and so that more functionality could be added to the program. \\

\noindent To evaluate the different genetic algorithm models, 4 scenarios provided by GECCO 2016 were used. The scenarios provided contained wind parameters, terrain description, and wind turbine parameters. First, experiments were run to find parameter values for the genetic algorithm suitable for the wind farm layout optimization problem. Second the different genetic algorithms were run with those settings. The genetic algorithms were compared on four different wind scenarios and the results for each scenario is presented, along with overall performance on all scenarios for each genetic algorithm. Finally, the results from the different simulations were compared and analyzed in order to to answer the research questions.


\section{Contributions}\label{section:introduction contributions}
The contribution to the field of wind farm layout optimization provided by this thesis is a comparison of different population distributed genetic algorithms and their ability to solving the problem. As far as the author knows, no attempt has been made in comparing population distributed genetic algorithms on the wind farm layout optimization problem, therefore this thesis can contribute with clearance as of which, if any, is the best choice for the given problem.\\

\noindent This thesis also contribute with an extensive survey of the state of the art within wind farm layout optimization and artificial intelligence approaches. The survey is provided in chapter \ref{chapter:relatedwork} and can be useful beyond the scope of this thesis for everyone who wants to get an overview of research within the field of wind farm layout optimization using genetic algorithms.\\


\section{Thesis Structure}\label{thesisstructure}
The thesis is divided into six chapters. Chapter \ref{chapter:background} contains an introduction to the wind farm layout optimization problem, a description of the simple genetic algorithm, and a description of each of the different population distributed genetic algorithms that will be implemented. Chapter \ref{chapter:relatedwork} is a survey of the state of the art within wind farm layout optimization using genetic algorithms and other methods. The chapter also contains a discussion/conclusion of related work. Chapter \ref{chapter:methodology} contains the methodology used to implement and test the genetic algorithms. In chapter \ref{chapter:results} the results are presented and discussed, and finally, the research questions are answered and discussed in chapter \ref{chapter:conclusion}.