\chapter{Introduction}\label{chapter:introduction}
This thesis is a contribution to the third edition of the wind farm layout optimization contest launched by the Genetic and Evolutionary Computation Conference (GECCO) 2016. The contest involves optimizing the number of turbines and turbine positioning in a wind farm with goal of producing maximum power to minimum cost using genetic algorithms. A wind farm simulator is provided by GECCO 2016, the focus of this thesis will therefore be on improving the genetic algorithm that will be used to search for the optimal wind farm layout by investigating the state of the art within wind farm layout optimization and genetic algorithms and experiment with variations of the genetic algorithm.\\

\noindent This chapter provides an introduction to this thesis. Section \ref{section:motivation&background} contains the background and motivation behind the thesis. Section \ref{section:goal&researchquestions} introduces the goal and research questions that will be investigated. In section \ref{section:researchmethod} a brief overview of the research method is given. Contributions is given in section \ref{section:contributions}, and an overview of the thesis structure is given in section \ref{thesisstructure}.


\section{Motivation and Background}\label{section:motivation&background}
Transitioning from non-renewable energy sources to renewable energy sources is one of the largest, if not the larges political challenge of today. Renewable energy is less polluting than non-renewable energy and should therefore be preferred. However, renewable energy sources make up only 21\% of the world's energy sources as of 2011 \citep{url1}. Wind turbine technology is a promising source of renewable energy. Wind turbine technology advances have led to wind turbines able to produce more energy to lower costs. Despite these improvements, wind turbines still produce less energy than predicted because of the wake effect; reduction in wind speed caused by turbines placed in front of other turbines \citep{Samorani}. For wind energy to become a bigger player in the world's energy sources, sophisticated methods for wind turbine placement in wind farms need to be developed so that each turbine produces as much energy as possible. \\

\noindent Wind turbine positioning is hard to optimize analytically. Fortunately, a wide variety of local search methods and bio-inspired methods have shown promising results, with genetic algorithms being the most popular method. As more advanced approaches to evaluate layouts has been developed, and more realistic constraints are introduced, more sophisticated genetic algorithms are required. To come up with more advanced genetic algorithms for solving the wind farm layout optimization problem, the annual Genetic and Evolutionary Computation Conference (GECCO) 2016, launched a competition where different contestants will provide their own implementation of a genetic algorithm \citep{url2}. The goal of the competition is to bring more realistic problems to algorithm developers, and to create an open source library useful beyond the scope of the competition. Wind parameters and evaluation mechanisms are provided by GECCO 2016, therefore, the focus of this thesis will be on optimizing the genetic algorithm. Still, some knowledge of wind turbines, wind farms, and wind- and power models are crucial to understanding this thesis and will therefore be introduced in chapter \ref{chapter:background}. \\

\noindent Greedy heuristics, simulated annealing search, ant colony algorithms, particle swarm optimization and genetic algorithms have all been used in solving the wind farm layout optimization problem, and these will all be reviewed in chapter \ref{chapter:relatedwork}. Turbine positioning has been improved by genetic algorithms for more than 20 years, each approach bringing something new to the field such as a new type of genetic algorithm, a more realistic environment, or combinations of genetic algorithms and other approaches. Many researchers have solved the wind farm layout optimization problem by implementing a population distributed genetic algorithms called the Island model, and it has shown promising results. However, as far as the author knows, no attempt has been made in implementing any of the other existing population distributed models. This fact is the main motivation behind this thesis, and has inspired the author to investigate the effect different population distributed algorithms can have on the wind farm layout optimization problem. Chapter \ref{chapter:background} contains a review of the population distributed genetic algorithm investigated in this thesis.


\section{Goal and Research Questions}\label{section:goal&researchquestions}
This section states the goal statement and research questions that will be investigated in this thesis. \\

\noindent \textbf{Goal statement}

\begin{quote}
\textit{The project goal is to investigate the advantages of using population distributed genetic algorithms in optimizing wind farm layout, i.e. solving the wind farm layout optimization problem.} \citep{Samorani}
\end{quote}

\noindent The performance of population distributed genetic algorithms will be studied and compared to the performance of a simple genetic algorithm (non population distributed genetic algorithm) in order to answer the first research question. In addition, each of the population distributed genetic algorithms will be compared against the others with the goal of answering the second research question. \\

\noindent \textbf{Research question 1}

\begin{quote}
\textit{Can distributed genetic algorithms improve the quality of the solution to the wind farm layout optimization problem as compared to simple genetic algorithm.}
\end{quote}

\noindent \textbf{Research question 2}

\begin{quote}
\textit{Which distributed genetic algorithm works best for the wind farm layout optimization problem? What properties are essential for its success?}
\end{quote}

\noindent Both research questions will be answered by testing the different distributed genetic algorithms in a wind farm simulator provided by GECCO 2016. The genetic algorithms will be implemented from scratch in order to give the author control of the environment and the possibility of adding functionality fast and easy. Each genetic algorithm will be tested on different wind scenarios which are also provided by the contest. Research question 1 will be answered by implementing 3 different types of population distributed genetic algorithms and compare their results with a simple genetic algorithm on different wind scenarios. Research question 2 will be tested the same way, and the results of each of the population distributed algorithms will be compared.


\section{Research Method}\label{section:researchmethod}
In order to answer the research questions, a genetic algorithm was implemented for this thesis. Even though a simple genetic algorithm were provided by GECCO 2016 along with a wind farm evaluator and wind scenarios, the choice was made to implement the algorithm from scratch so that the population distributed genetic algorithms could be added easily, and so that more functionality could be added to the program. \\

\noindent To evaluate the different genetic algorithms, scenarios provided by GECCO 2016 were used. The contest provided the contestants with realistic site scenarios which included wind speed from different angels, terrain description such as which areas are possible to place wind turbines in and which contains obstacles, along with other wind turbine parameters. The genetic algorithms were compared on different scenarios, but also on averaged performance over the different wind scenarios. Different genetic algorithm settings were tested in order to find the best settings to run the experiments on.\\

\noindent Finally, the results from the different simulations were compared and analyzed in order to point out which features are important in solving the wind farm layout optimization problem, and what properties that are key for success. \textcolor{red}{Fix!}


\section{Contributions}\label{section:contributions}
The contribution to the field of wind farm layout optimization using genetic algorithms provided by this thesis is a comparison of different population distributed genetic algorithms and their effect on solving the problem. As far as the reader knows, no attempt has been made in comparing population distributed genetic algorithms on the wind farm layout optimization problem, therefore this thesis can contribute with clearance as of which, if any, is the best choice.\textcolor{red}{Fix!}


\section{Thesis Structure}\label{thesisstructure}
The thesis is divided into four chapters. Chapter \ref{chapter:background} contains an introduction to the wind farm layout optimization problem, a description of how the genetic algorithm works, and a description of each of the population distributed genetic algorithms that will be implemented. Chapter \ref{chapter:relatedwork} is a survey of the state of the art within wind farm layout optimization, one section describing approaches using genetic algorithms, and one section describing other approaches. Chapter \ref{chapter:technical} is a description of the application user interface provided by GECCO 2015 that will be extended with different distributed genetic algorithm implementations. It also contains test simulations, and a discussion of future work. \textcolor{red}{Fix!}