\chapter{Results and Discussion}\label{chapter:results}
This chapter includes presentation and analysis of the results obtained in this thesis. Section \ref{section:parameter settings} presents results from testing different parameter settings for the genetic algorithm, and briefly discusses each results. Section \ref{section:results} presents the main results obtained when running the different population distributed genetic algorithms. Section \ref{section:discussion} contains a discussion and comparison of the results obtained in section \ref{section:results}.


\section{Parameter settings}\label{section:parameter settings}
Parameter settings are crucial for obtaining good results with the genetic algorithm. In order to find the right settings, simulations were run to find the best adult selection method, parent selection method, crossover method, crossover rate and mutation rate for the given problem. Even though it would take much shorter time and less effort to test these settings on a toy problem such as One Max, the decision was made to test them on the real problem. The reason behind this decision is that optimizing the parameters and selection mechanism on a toy problem is problematic. The wind farm layout optimization problem is extremely complex and hard to solve even with genetic algorithms because the relative positions of the wind turbines are extremely important, suggesting that the parameters and selection mechanisms that gives best results in wind farm layout optimization is might not the same that will be optimal for a simple toy problem. It is important to note that even though effort was made to find the right settings for the genetic algorithm,  it is impossible to obtain the optimal ones. Just imagine trying the test every single value for the continuous parameter crossover rate, just that would be impossible! Therefore, the values that are tested for each parameter is based on the authors previous knowledge and experience with genetic algorithms. \textcolor{red}{Should I say something about evaluation time in this section?}\\

\begin{table}
\centering
\caption{Values that were kept fixed while one by one were optimized to find the correct settings for the find farm layout optimization problem.}
\label{table:fixed settings}
\begin{tabular}{l|l}
\textbf{Parameter} & \textbf{Value} \\ 
\hline 
Wind scenario & 00.xml \\ 
Evaluator & KusiakLayoutEvaluator \\ 
Population size & 100 \\  
Generations & 100 \\ 
Elitism & true \\  
Flip mutation rate & 0.01 \\ 
Inversion mutation rate & 0.0 \\ 
Interchange mutation rate & 0.0 \\ 
Parent selection & Tournament selection \\ 
Tournament size & 5 \\ 
Epsilon & 0.0 \\  
Crossover method & Uniform crossover \\ 
Crossover rate & 0.9 \\ 
\end{tabular} 
\end{table}

\noindent For each simulation, one of the parameters were tested while the others were kept fixed as shown in table \ref{table:fixed settings}. As can be seen in this table, other variables such as wind scenario, population size, number of generations, whether elitism should be used and mutation rate for interchange mutation and inversion mutation could also be tested. However, since evaluation of each farm takes half a day, even on an 8 core computer running in parallel, the time frame of this thesis prevented the author from investigation these variables.. These variables are set using the authors previous experience with genetic algorithms. Testing different parameter values on a single wind scenario might lead to values that are tailored for the given scenario and that are not as well suited for some of the others. However, there is no time to test each value for every single wind scenario, and this should not make that much of a difference. Population size is a value that influence the performance of the genetic algorithm largely: Greater population size increases the probability of finding the global best solutions. The population size was set to 100 for two reasons. First of all, a population size of 100 is large enough so that many different solutions can be explored. Second, when the population size is kept at 100, the adult selection method \textit{overproduction} will produce twice as many children, so that 200 individuals has to be evaluated for each generation something that will double the evaluation time, the step that already is the bottleneck of the algorithm. The number of generations was kept at 100 for each run. As mentioned above, evaluation is time consuming, and letting the algorithm run for more than 100 generations for each value for each parameter tested would not be possible within the time frame of this thesis. However, as can be seen on the graphs in the following sections, the fitness is starting to flat out after a 100 generations indicating that 100 generations are sufficient for the purpose of setting the parameter values. Elitism was set to true for every run without testing, meaning that the best individual of each generation will survive. This decision is based on the authors previous experience with genetic algorithm where experiments has shown that elitism leads to better results because the best individual is not lost due to coincidences. Different mutation methods were implemented for the genetic algorithm, but only the flip mutation rate was tested to find the best value. Usually, flip mutation is the only mutation method used with genetic algorithms and it is therefore also used as the main method for of mutation in this thesis. Inversion mutation and interchange mutation are implemented, but they will only happen at rare occasions to introduce more randomness to the algorithm. In the main simulations, they will be assigned extremely low probabilities so that their occurrence is so rare that they will introduce too much randomness, but hopefully introduce occasional jumps to different solutions so that the algorithm do not get completely stuck in a local optimal solution. 


\subsection{Adult Selection}
Figure \ref{plot:adult selection methods} shows the results of running the genetic algorithm with the three different adult selection methods. Figure \ref{plot:full generational replacement}, \ref{plot:generational mixing}, and \ref{plot:overproduction} shows the results for full generational replacement, generational mixing and overproduction respectively. \\

\noindent The results show that full generational replacement ends up with worse fitness than both generational mixing and overproduction, and that overproduction is the method that is able to achieve the best fitness of the three. These results agrees with the authors previous experience. Note that with generational mixing, only the solutions in the new child population that are better than those in the previous child population are allowed to enter the adult pool. This means that for each generation the fitness of the new child population can never be worse than the fitness of the previous child population. If every new solution generated is worse than those from the previous generation, the previous generation is given another attempt to reproduce and the fitness from one generation to another is left unchanged. However, if one or more of the new solutions are better than those in the previous generation they will be allowed into the adult pool, and the new adult pool will have better fitness than the previous. Full generation replacement replace the entire population, even thought it will lead to worse average fitness than that of the previous generation. Regardless, full generational replacement is not destined to do worse than generational mixing, it could be the case that even thought the average fitness might go down from one generation to another, it might lead to better results in the end because picking the seemingly best solution as generational mixing does will not always lead to better fitness in the end. However, on this problem, the properties of generational mixing seems to be advantageous. Overproduction will, as full generational replacement, replace the entire previous adult pool with new individuals. However, with overproduction, the probability that the new generation will have better average fitness is much higher since it has twice as many children to choose from. This means that even though it is not guaranteed, it is likely that the new adult population will have higher fitness than the previous one. Still, as mentioned with generational mixing, taking greedy decisions like these might lead to sub optimal results at the end, and there is no guarantee that overproduction leads to better results than full generational replacement on every problem, but it does for the wind farm layout optimization problem. These observations will be discussed in depth in section \ref{section:discussion}.


\begin{figure}[h!]
    \centering
    \begin{subfigure}[b]{0.31\textwidth}
        \includegraphics[width=\textwidth]{images/plots/"adult selection"/"full generational replacement"}
        \caption{}
        \hfill
        \label{plot:full generational replacement}
    \end{subfigure}
    ~
    \begin{subfigure}[b]{0.31\textwidth}
        \includegraphics[width=\textwidth]{images/plots/"adult selection"/"generational mixing"}
        \caption{}
        \hfill
        \label{plot:generational mixing}
    \end{subfigure}
    ~
    \begin{subfigure}[b]{0.31\textwidth}
        \includegraphics[width=\textwidth]{images/plots/"adult selection"/"overproduction"}
        \caption{}
        \hfill
        \label{plot:overproduction}
    \end{subfigure}
    \caption{Adult selection methods: (a) Full generational replacement, (b) generational mixing and (c) overproduction. Each plot shows the averaged results obtained when running each scenario 10 times.}
    \label{plot:adult selection methods}
\end{figure}


\subsection{Parent Selection}
Figure \ref{plot:parent selection} shows the results for running the genetic algorithm with parent selection methods roulette wheel is sub-figure \ref{plot:roulette wheel}, and tournament selection in sub-figures \ref{plot:tournament size 5} - \ref{plot:tournament size 25}. Tournament selection is run with five different values for the variable \textit{tournament size}. Clearly, roulette wheel do not work well for this problem. Roulette wheel assigns each individual a probability of being selected proportional to its fitness, since there is not a large different in fitness of the different individuals the selection becomes almost random. With tournament selection on the other hand, this problem is taken care of because it finds the best individuals out of those competing, event though there is not much different in their fitness.\\


\noindent As can be seen in the figure, the fitness gets better as the variable \textit{tournament size} increases. However, a larger tournament size than 25 is not tested. The reason for this is that 25 is already a very large tournament size. With 25 \% of the individuals competing in every tournament the population will not have much time to explore different solutions because the best individuals will take over the entire population extremely fast. As can be seen in the figure, a tournament size of 25 is not significantly better than a tournament size of 20, and therefore 20 is chosen as the final tournament size to slow down the take-over time. \\


\noindent As mentioned before, \textit{epsilon} is the probability that the selected individual is selected from a random position in the adult pool instead of by tournament selection. Figure \ref{plot:epsilon} shows the result for testing the values 5 \%, 10 \%, and 15 \%. As can be seen in the figure, varying epsilon does not have a significant impact on the fitness on the wind farm layout optimization problem.\\


\begin{figure}[h!]
    \centering
    \begin{subfigure}[b]{0.31\textwidth}
        \includegraphics[width=\textwidth]{images/plots/"parent selection"/"roulette wheel"}
        \caption{}
        \hfill
        \label{plot:roulette wheel}
    \end{subfigure}
    ~
    \begin{subfigure}[b]{0.31\textwidth}
        \includegraphics[width=\textwidth]{images/plots/"parent selection"/"tournament size 5"}
        \caption{}
        \hfill
        \label{plot:tournament size 5}
    \end{subfigure}
    ~
       \begin{subfigure}[b]{0.31\textwidth}
        \includegraphics[width=\textwidth]{images/plots/"parent selection"/"tournament size 10"}
        \caption{}
        \hfill
        \label{plot:tournament size 10}
    \end{subfigure}
    ~
       \begin{subfigure}[b]{0.31\textwidth}
        \includegraphics[width=\textwidth]{images/plots/"parent selection"/"tournament size 15"}
        \caption{}
        \hfill
        \label{plot:tournament size 15}
    \end{subfigure}
    ~
       \begin{subfigure}[b]{0.31\textwidth}
        \includegraphics[width=\textwidth]{images/plots/"parent selection"/"tournament size 20"}
        \caption{}
        \hfill
        \label{plot:tournament size 20}
    \end{subfigure}
    ~
    \begin{subfigure}[b]{0.31\textwidth}
        \includegraphics[width=\textwidth]{images/plots/"parent selection"/"tournament size 25"}
        \caption{}
        \hfill
        \label{plot:tournament size 25}
    \end{subfigure}
    \caption{Parent selection methods: (a) Roulette wheel, (b) tournament selection, tournament size 5 (c) tournament selection, tournament size 10, (d) tournament selection, tournament size 15, (e) tournament selection, tournaments size 20, and (f) tournament selection, tournament size 25. Each result is an average over 10 runs.}
    \label{plot:parent selection}
\end{figure}


\begin{figure}[h!]
    \centering
    \begin{subfigure}[b]{0.31\textwidth}
        \includegraphics[width=\textwidth]{images/plots/epsilon/"tournament size 20"/"epsilon 0,05"}
        \caption{}
        \hfill
        \label{plot:epsilon 0.05}
    \end{subfigure}
    ~
    \begin{subfigure}[b]{0.31\textwidth}
        \includegraphics[width=\textwidth]{images/plots/epsilon/"tournament size 20"/"epsilon 0,10"}
        \caption{}
        \hfill
        \label{plot:epsilon 0.10}
    \end{subfigure}
    ~
    \begin{subfigure}[b]{0.31\textwidth}
        \includegraphics[width=\textwidth]{images/plots/epsilon/"tournament size 20"/"epsilon 0,15"}
        \caption{}
        \hfill
        \label{plot:epsilon 0.15}
    \end{subfigure}
    \caption{Different epsilon values: (a) Epsilon 0.05, (b) epsilon 0.10, and (c) epsilon 0.15. The tournament size was kept fixed at 20 since that was proven to give the best results. Each plot is an average over 10 runs.}
    \label{plot:epsilon}
\end{figure}


\subsection{Crossover Methods}
Figure \ref{plot:crossover methods} show the results when running the genetic algorithm with single point crossover, two point crossover, and uniform crossover as can be seen in sub figures \ref{plot:single point crossover}, \ref{plot:two point crossover}, and \ref{plot:uniform crossover} respectively. The experiment show that no crossover method is significantly better than the others in solving the wind farm layout optimization problem. This result is unexpected. Intuitively, one would expect uniform crossover to perform worse because it mixes up the relative positions of the wind turbines more than single point crossover and two point crossover. However, the results contradicts this hypothesis. \\


\begin{figure}[h!]
    \centering
    \begin{subfigure}[b]{0.31\textwidth}
        \includegraphics[width=\textwidth]{images/plots/"crossover method"/"single point crossover"}
        \caption{}
        \hfill
        \label{plot:single point crossover}
    \end{subfigure}
    ~
    \begin{subfigure}[b]{0.31\textwidth}
        \includegraphics[width=\textwidth]{images/plots/"crossover method"/"two point crossover"}
        \caption{}
        \hfill
        \label{plot:two point crossover}
    \end{subfigure}
    ~
    \begin{subfigure}[b]{0.31\textwidth}
        \includegraphics[width=\textwidth]{images/plots/"crossover method"/"uniform crossover"}
        \caption{}
        \hfill
        \label{plot:uniform crossover}
    \end{subfigure}
    \caption{Crossover methods averaged over 10 runs: (a) Single point crossover, (b) two point crossover and (c) uniform crossover.}
    \label{plot:crossover methods}
\end{figure}


\subsection{Crossover Rate}
Figure \ref{plot:crossover rates} displays the results of running the genetic algorithm with different crossover rates. As can be seen in the figure, the crossover rate do not have a large impact on the fitness for this problem. A crossover rate below 0.4 gives ends up with slightly worst fitness than a crossover rate above 0.6. However, it is not obvious which crossover rate is best out of those presented in sub figures \ref{plot:crossover rate 0.6}, \ref{plot:crossover rate 0.8}, and \ref{plot:crossover rate 1.0}. The differences in the results obtained when running the algorithm with these crossover rates are not significant, and it can not be stated that one is better than the others based on these results. It can only be stated that a crossover rate above 0.6 should be used. \\


\begin{figure}[h!]
    \centering
    \begin{subfigure}[b]{0.31\textwidth}
        \includegraphics[width=\textwidth]{images/plots/"crossover rate"/"crossover rate 0,0"}
        \caption{}
        \hfill
        \label{plot:crossover rate 0.0}
    \end{subfigure}
    ~
    \begin{subfigure}[b]{0.31\textwidth}
        \includegraphics[width=\textwidth]{images/plots/"crossover rate"/"crossover rate 0,2"}
        \caption{}
        \hfill
        \label{plot:crossover rate 0.2}
    \end{subfigure}
    ~
       \begin{subfigure}[b]{0.31\textwidth}
        \includegraphics[width=\textwidth]{images/plots/"crossover rate"/"crossover rate 0,4"}
        \caption{}
        \hfill
        \label{plot:crossover rate 0.4}
    \end{subfigure}
    ~
       \begin{subfigure}[b]{0.31\textwidth}
        \includegraphics[width=\textwidth]{images/plots/"crossover rate"/"crossover rate 0,6"}
        \caption{}
        \hfill
        \label{plot:crossover rate 0.6}
    \end{subfigure}
    ~
       \begin{subfigure}[b]{0.31\textwidth}
        \includegraphics[width=\textwidth]{images/plots/"crossover rate"/"crossover rate 0,8"}
        \caption{}
        \hfill
        \label{plot:crossover rate 0.8}
    \end{subfigure}
    ~
    \begin{subfigure}[b]{0.31\textwidth}
        \includegraphics[width=\textwidth]{images/plots/"crossover rate"/"crossover rate 1,0"}
        \caption{}
        \hfill
        \label{plot:crossover rate 1.0}
    \end{subfigure}
    \caption{Crossover rates average over 10 runs: (a) Crossover rate 0.0, (b) crossover rate 0.2, (c) crossover rate 0.4, (d) crossover rate 0.6, (e) crossover rate 0.8, and (f) crossover rate 1.0.}
    \label{plot:crossover rates}
\end{figure}


\subsection{Mutation Rate}
In figure \ref{plot:mutation rate}, the effect of varying the mutation rate can be seen. As the figure shows, if the mutation rate is very low as shown in sub-figure \ref{plot:mutation rate 0.0001}, the population is clearly not able to find a good solution. Mutation means adding or removing a turbine. As the sub-figure show, with a mutation rate that is too low the genetic algorithm will not be able to add or remove enough turbines and therefore the population ends up in a local minima. On the other hand, when the mutation rate gets too high, as shown in sub-figure \ref{plot:mutation rate 0.01}, the population is able to explore many different solutions something that increases its chance of finding the global minima. However, since mutation is performed too frequently, it is not able to stay there even if it finds the global optimal solution. These results show that mutation rate largely impacts the fitness, and that a mutation rate of 0.001 clearly gives the best results.


\begin{figure}[h!]
    \centering
      \begin{subfigure}[b]{0.31\textwidth}
        \includegraphics[width=\textwidth]{images/plots/"mutation rate"/"mutation rate 0,0001"}
        \caption{}
        \hfill
        \label{plot:mutation rate 0.0001}
    \end{subfigure}
    ~
      \begin{subfigure}[b]{0.31\textwidth}
        \includegraphics[width=\textwidth]{images/plots/"mutation rate"/"mutation rate 0,0005"}
        \caption{}
        \hfill
        \label{plot:mutation rate 0.0005}
    \end{subfigure}
    ~
    \begin{subfigure}[b]{0.31\textwidth}
        \includegraphics[width=\textwidth]{images/plots/"mutation rate"/"mutation rate 0,001"}
        \caption{}
        \hfill
        \label{plot:mutation rate 0.001}
    \end{subfigure}
    ~
    \begin{subfigure}[b]{0.31\textwidth}
        \includegraphics[width=\textwidth]{images/plots/"mutation rate"/"mutation rate 0,005"}
        \caption{}
        \hfill
        \label{plot:mutation rate 0.005}
    \end{subfigure}
    ~
    \begin{subfigure}[b]{0.31\textwidth}
        \includegraphics[width=\textwidth]{images/plots/"mutation rate"/"mutation rate 0,01"}
        \caption{}
        \hfill
        \label{plot:mutation rate 0.01}
    \end{subfigure}
    \caption{Mutation rate averaged over 10 runs: (a) Mutation rate 0.0001, (b) mutation rate 0.0005 and (c) mutation rate 0.001, (d) mutation rate 0.005 and (e) mutation rate 0.01s.}
    \label{plot:mutation rate}
\end{figure}


\section{Results}\label{section:results}


\subsection{Final Parameter Values}


\noindent Table \ref{table:final parameter settings master slave model} provides an overview of the parameter values used when running the Master/Slave model. The crossover method, crossover rate, mutation rate, adult selection mechanism, parent selection mechanism and parent selection parameters are those that proved to work best for the Master/Slave model when solving the wind farm layout optimization problem as shown in section \ref{section:parameter settings}. The population size was set to 100 and the number of generations to 200. A population size of 100 is quite small, but since overproduction is the selected adult selection method a population size of 100 leads to evaluation of 200 individuals for each generation. Since evaluation is the bottle neck for wind farm layout optimization, a larger population size would not be possible to evaluate for 200 generations within the time limits of this thesis. A possibility could be to pick a larger population size and a smaller number of generations, but, since the objective of this thesis is to explore population distributed genetic algorithms a large number of generations is crucial. Population distributed genetic algorithms need more time to find good solutions because they spend more time than simple genetic algorithms on exploring different parts of the search space. \\


\begin{table}[h!]
\centering
\caption{Parameter values used for the master slave model.}
\label{table:final parameter settings master slave model}
\begin{tabular}{l|l}
\textbf{Parameter} & \textbf{Value} \\ 
\hline 
Population size & 100 \\  
Generations & 200 \\ 
Crossover method & Single point crossover \\ 
Crossover rate & 0.9 \\ 
Elitism & True \\ 
Flip mutation rate & 0.001 \\ 
Inversion mutation rate & 0.000001 \\ 
Interchange mutation rate & 0.000001 \\ 
Adult selection mechanism & Overproduction \\ 
Parent selection mechanism & Tournament selection \\ 
Tournament size & 20\% of population size\\ 
Epsilon & 0.1 \\ 
\end{tabular} 
\end{table}


\noindent Table \ref{table:final parameter settings island model} shows the parameters that are specific for the Island model. On each Island, the parameter values from table \ref{table:final parameter settings master slave model} are used and they are therefore not restated. The parameter \textit{deme size}, the number of individuals on each Island, is set to 26 resulting in a total population size of 104, not 100 as in table \ref{table:final parameter settings master slave model}. The reason behind this is that the implementation requires a population size of even numbers on each Island, this is not a problem since a few additional individuals will not be enough to impact the final results. The parameter \textit{migration rate} was set to 2, meaning that only 7 \% of the individuals on an Island are replaced by arriving individuals. This number was set to be low so that the new individuals would not be able to take over the new population too fast. The parameter \textit{deme count} is set to 4 and the topology is circular as presented in figure \ref{figure:topology island model} in chapter \ref{chapter:methodology}. In order to let the populations on the different Islands explore different parts of the search space 20 generations are run between each migration. Migration is performed 10 times so that the total number of generations becomes 200 as for the Master/Slave model. \\


\begin{table}[h!]
\centering
\caption{Parameter values used for the Island model.}
\label{table:final parameter settings island model}
\begin{tabular}{l|l}
\textbf{Parameter} & \textbf{Value} \\ 
\hline 
Deme size & 26 \\
Total population size & 104 \\  
Deme count & 4 \\
Migration rate & 2 \\
Number of migrations & 10 \\ 
Migration interval & 20 \\
Topology & Circular (figure \ref{figure:topology island model}) \\
\end{tabular}
\end{table}


\noindent As can be seen in table \ref{table:final parameter settings pool model}, the Cellular model is run with population size of 225. At a first glance this might seem like an odd decision since the models above has a population size of about 100 individuals, but the reason is simple. For the Master/Slave model, and Island model overproduction is used as the adult selection technique because it was proven the best technique in section \ref{section:parameter settings}. For the Cellular model however, it does not make sense to use overproduction because of the way the grid is updated. Each new individual is generated to occupy a particular spot in the grid, because it has its own area for which its parents can be chosen from. Since each new individual is mapped to one position in the grid, overproduction does not make sense to use and it therefore not used for the Cellular model. Because of this, the population size is set to approximately 200 individuals, so that the Cellular model performs the same number of evaluations as the other models each generation, and the comparison becomes more fair than with a population size of 100. The reason why the population size is 225 and not 200 is because a quadratic grid is used to distribute the production and 225 is equal to 15$^{2}$. As explained in the methodology chapter, the topology used is a simple square containing nine individuals. The individual in the middle square can only be replaced by individuals generated by recombining individuals within the square grid. The decision to make the neighborhood this small is that it will give different solutions the opportunity to dominate different areas of the grid, giving the algorithm the time to explore.\\


\begin{table}
\centering
\caption{Parameter values used for the Cellular model.}
\label{table:final parameter settings cellular model}
\begin{tabular}{l|l}
\textbf{Parameter} & \textbf{Value} \\ 
\hline 
Population size & 225 \\
Topology & Square (figure \ref{figure:cellular model topology}) \\  
\end{tabular}
\end{table}


\noindent Table \ref{table:final parameter settings pool model} shows the parameter values used by the pool model that differs from those in table \ref{table:final parameter settings master slave model}. As for the Cellular model, the population size is set to 200 so that 200 evaluations are performed for each of the 200 generations. As for the Cellular model, the individuals of the Pool model occupy specific positions. Child generation consist of producing one new individual for each position the the corresponding worker is responsible for. If overproduction was used, it would be no clear mapping between new child individual and old individual because twice as many individuals as position would be generated, this does not make sense for the Pool model. Because of this a population size of 200 individuals where used so that 200 evaluations are performed for each generation as for the other models.\\


\begin{table}[h!]
\centering
\caption{Parameter values used for the Pool model.}
\label{table:final parameter settings pool model}
\begin{tabular}{l|l}
\textbf{Parameter} & \textbf{Value} \\ 
\hline 
Population size & 200 \\  
Number of workers & 4 \\ 
\end{tabular}
\end{table}


\noindent In summary, each algorithm run for 200 generations, and approximately 200 individuals are evaluated for each generation. This makes sure that every model gets a fair amount of processor time.\\


\subsection{Measurement Score}


The models are tested on 4 different wind scenarios shown in table \ref{table:scenario properties}. As can be seen, the first two scenarios contain obstacles and the last two do not. \\


\begin{table}[h!]
\centering
\caption{Properties of the different scenarios used to test the different models.}
\label{table:scenario properties}
\begin{tabular}{l|l}
\textbf{Scenario} & \textbf{Properties} \\
\hline
00.xml            & No obstacles. \\
05.xml            & No obstacles. \\
obs00.xml         & Obstacles. \\
obs05.xml         & Obstacles. \\
\end{tabular}
\end{table}


\noindent For each scenario, the performance of the models are measured using the measures shown in table \ref{table:overview}. Each of these measures will be presented in a plot for each model for each scenario. The fitness presented in table \ref{table:overview} is the \textit{fitness} of the best individual in the population for each generation. The \textit{efficiency} is the total amount of energy produced by the best individual, out of the upper bound on energy production; energy produced without the existence of wake effect. \textit{Cost} (USD) is the total cost of the best individual including turbine costs, substation cost, and yearly operating cost. Power is the yearly power output in (kWh). Number of turbines is the total number of turbines for the best individual. \textcolor{red}{Should I show this with the fitness function, how it is partitioned to find the different results. Ask Jean.}\\


\begin{table}[h!]
\centering
\caption{For each scenario, results are measured in these measurements.}
\label{table:overview}
\centering
\begin{tabular}{l|l}
\textbf{Results measured}   & \textbf{Description} \\
\hline
Fitness                     & Fitness of the best individual \\
Efficiency                  & Amount of power produced out of maximum \\
Cost                        & Total cost (USD)\\
Power                       & Yearly power output (kWh) \\
Number of turbines          & Total number of turbines
\end{tabular}
\end{table}


\noindent In the sub sections below will the results for each scenario be presented. In addition, a final plot will show the performance of the different models averaged over all 4 scenarios.\textcolor{red}{More?}\\


\subsection{Results Scenario 00}
The results when all the models were run on scenario 00.xml is shown in figure \ref{plot:scenario 00}. The fitness plot is shown in sub-figure \ref{plot:fitness plot scenario 00}. The Master/Slave model is able to get the best fitness, the Pool model comes in second, quite close to the Master/Slave model, the Island model comes in third, and the Cellular model comes in forth, clearly not able to keep up with the other models.\\ 


\noindent Sub-figure \ref{plot:efficiency plot scenario 00} shows the performance of the different models in terms of efficiency. Efficiency is a very interesting measure, since it shows the models' ability to live up to their potential. As will be discussed below, the Master/Slave model and Pool model end up with a different number of turbines. Interestingly, within their chosen number of turbines, both models are able to obtain almost the same efficiency, showing that they are both good at optimizing turbine positions.\\


\noindent As shown in sub-figures \ref{plot:cost plot scenario 00} and \ref{plot:turbines plot scenario 00}, the plots for cost and number of turbines have identical shape. This makes sense off course since the cost increase and decrease with the number of turbines. The plots show that the Master/Slave model end up with a larger number of turbines than the other models, which all end up with the same number of turbines. The Master/Slave model, the Pool model and the Cellular model all decide on the number of turbines before having evolved for 20 generations, while the Island model is able to explore different solutions until around generation 50, when it stabilized on the same solution as the Pool- and Cellular model. The Island model cost plot has a different shape than the other plots. While the other models slowly go from one solution to another, the Island model jumps from one part of the solution space to another. These observations make sense, because, these jumps occur at generation 20 and 40, the generations where the first and second migration takes place. 

Between generation 20 and 40, the Island model explores a solution with the same number of turbines as the Master/Slave model, a solution that could be the global minima. However, the Island model is not able to stay in this minima, because it observes that jumping to a solution with fewer turbines seems more promising. If the migration interval had been longer than 20 generations, the Island model might have been able to optimize turbine positions in the global minima so that it would discover the true potential of the solution before leaving it for a sub-optimal solution.\\


\noindent The performance of the different models can be viewed in terms of power in sub-figure \ref{plot:power plot scenario 00}. Power is also closely related to the number of turbines. Clearly a solution with more turbines leads to higher energy output. However, as the Island model power plot shows, solutions with high power production might not be prioritized because the cost becomes too high so that the total fitness becomes worse. After the models have stabilized on a particular number of turbines, the power is slowly increasing. This is because the turbines are slowly moved into positions with less wake loss. \\


\noindent In summary, the optimization can be viewed as consisting of two phases. The first phase is the search for the optimal number of turbines, while the second phase consist of moving turbines around in order to reduce wake loss. The Island model stays in phase one longer than the other models. Still, the Master/Slave model is the only model able to find what could be the global optimal solution in terms of the number of turbines. \\

\begin{figure}[h!]
    \centering
      \begin{subfigure}[b]{0.31\textwidth}
        \includegraphics[width=\textwidth]{images/plots/Plots/"scenario 00"/"best"}
        \caption{Fitness}
        \hfill
        \label{plot:fitness plot scenario 00}
    \end{subfigure}
    ~
      \begin{subfigure}[b]{0.31\textwidth}
        \includegraphics[width=\textwidth]{images/plots/Plots/"scenario 00"/"efficiency"}
        \caption{Efficiency}
        \hfill
        \label{plot:efficiency plot scenario 00}
    \end{subfigure}
    ~
    \begin{subfigure}[b]{0.31\textwidth}
        \includegraphics[width=\textwidth]{images/plots/Plots/"scenario 00"/"cost"}
        \caption{Cost}
        \hfill
        \label{plot:cost plot scenario 00}
    \end{subfigure}
    ~
    \begin{subfigure}[b]{0.31\textwidth}
        \includegraphics[width=\textwidth]{images/plots/Plots/"scenario 00"/"power"}
        \caption{Power}
        \hfill
        \label{plot:power plot scenario 00}
    \end{subfigure}
    ~
    \begin{subfigure}[b]{0.31\textwidth}
        \includegraphics[width=\textwidth]{images/plots/Plots/"scenario 00"/"turbines"}
        \caption{Turbies}
        \hfill
        \label{plot:turbines plot scenario 00}
    \end{subfigure}
    \caption{Scenario 00.xml averaged over 10 runs: (a) Fitness plot, (b) efficiency plot, (c) cost plot, (d) power plot, and (e) number of turbines.}
    \label{plot:scenario 00}
\end{figure}


\subsection{Results Scenario 05}
The results for running the different models on scenario 05.xml is shown in figure \ref{plot:scenario 05}. The fitness plot is shown in sub-figure \ref{plot:fitness plot scenario 05}. As for scenario 00.xml, the Master/Slave model obtains the best fitness closely followed by the Pool model. The Island model comes in third again, and the Cellular model is not able to keep up with the other models on this scenario either. After 200 generations the Cellular model has a fitness which each of the other models had beaten already at generation 10.\\


\noindent The efficiency plot is shown in sub-figure \ref{plot:efficiency plot scenario 05}. As before, the Master/Slave model and Pool model is able to find a solution with approximately the same efficiency. The Island model is a few steps behind, while the efficiency of the Cellular model is increasing extremely slowly. \\


\noindent Sub-figures \ref{plot:cost plot scenario 05} and \ref{plot:turbines plot scenario 05} shows cost plot and number of turbines plot respectively. As can be seen, the Master/Slave model and Pool model does not explore different number of turbines, they both stabilize on a given number after less than 10 generations. The Cellular model explores for about 15-20 generations before it decide on the number of turbines. From the sub-figures it looks like the Island model explores for a little less than 50 generations, but if one takes a closer look, it actually does a small jump to a solution with fewer turbines at about generation 80. However, it is not able to stay in this solutions for longer than about 5 generations. As for scenario 00.xml, the Master/Slave model ends up with a higher number of turbines than the other 3 models. It seems like the Master/Slave model has found the global minima, or at least a better local minima than the other models, something that might explain why the Pool model is not able to get a fitness as good as the Master/Slave model.\\


\noindent Power is closely related to the number of turbines, and as can be seen in sub-figure \ref{plot:power plot scenario 05}, the Master/Slave model finds the solution that is able to produce most energy. The Pool Model and Island model both end up with the same number of turbines, and as can be seen the amount of power that they are able to produce is similar. The Pool model produces a little more power because it is able to find better positions for the turbines as can be seen since it beats the Pool model in fitness and efficiency. \\


\begin{figure}[h!]
    \centering
      \begin{subfigure}[b]{0.31\textwidth}
        \includegraphics[width=\textwidth]{images/plots/Plots/"scenario 05"/"best"}
        \caption{Fitness}
        \hfill
        \label{plot:fitness plot scenario 05}
    \end{subfigure}
    ~
      \begin{subfigure}[b]{0.31\textwidth}
        \includegraphics[width=\textwidth]{images/plots/Plots/"scenario 05"/"efficiency"}
        \caption{Efficiency}
        \hfill
        \label{plot:efficiency plot scenario 05}
    \end{subfigure}
    ~
    \begin{subfigure}[b]{0.31\textwidth}
        \includegraphics[width=\textwidth]{images/plots/Plots/"scenario 05"/"cost"}
        \caption{Cost}
        \hfill
        \label{plot:cost plot scenario 05}
    \end{subfigure}
    ~
    \begin{subfigure}[b]{0.31\textwidth}
        \includegraphics[width=\textwidth]{images/plots/Plots/"scenario 05"/"power"}
        \caption{Power}
        \hfill
        \label{plot:power plot scenario 05}
    \end{subfigure}
    ~
    \begin{subfigure}[b]{0.31\textwidth}
        \includegraphics[width=\textwidth]{images/plots/Plots/"scenario 05"/"turbines"}
        \caption{Turbies}
        \hfill
        \label{plot:turbines plot scenario 05}
    \end{subfigure}
    \caption{Scenario 05.xml averaged over 10 runs: (a) Fitness plot, (b) efficiency plot, (c) cost plot, (d) power plot, and (e) number of turbines.}
    \label{plot:scenario 05}
\end{figure}


\subsection{Results Scenario obs00.xml}
Figure \ref{plot:scenario obs 00} shows the results obtained when running the model on scenario obs00.xml. As for scenairo 00.xml and 05.xml, the Master/Slave model obtains the best fitness, the Pool model the second best, the Island model third best and the Cellular model the fourth best, as can be seen in sub-figure \ref{plot:fitness plot scenario obs 00}.\\


\noindent As for efficiency, the results are similar as those observed for scenario 00.xml and scenario 05.xml as can be seen in sub-figure \ref{plot:efficiency plot scenario obs 00}. Both the Pool model and Island model are able to optimize their solutions so that they produce more than 84.5 \% of the wake free energy. The Island model however, is only able to produce just above 83.5 \% of the wake free energy.\\


\noindent As shown in sub-figures \ref{plot:cost plot scenario obs 00} and \ref{plot:turbines plot scenario obs 00} the Island- and Cellular model end up with the same number of turbines, while the Master/Slave- and Pool model end up with different number of turbines. The Island- and Cellular model ends up at the solution with the highest number of turbines and the Pool models finds the solution with fewest turbines. This explains why the Island model scored so badly on efficiency, it finds out that adding turbines leads to higher power production and therefore better fitness and therefore it ends up in a sub-optimal solution. \\


\noindent The Island model finds the solutions which produces most power, a reasonable result since no other model ends up with more turbines. However, the Island model is closely followed by the Master/Slave model even though it finds a solution with fewer turbines. This shows that the Master/Slave model is way better at optimizing the turbine positions. On problematic observation is that the Cellular model ends up with a solution that produces less power than all the other models even thought it ends up with a solution with the same number of turbines as the Island model. This results show that the Cellular model is not able to find a solution that position the turbines well within 200 generations. \\


\begin{figure}[h!]
    \centering
      \begin{subfigure}[b]{0.31\textwidth}
        \includegraphics[width=\textwidth]{images/plots/Plots/"scenario obs 00"/best}
        \caption{Fitness}
        \hfill
        \label{plot:fitness plot scenario obs 00}
    \end{subfigure}
    ~
      \begin{subfigure}[b]{0.31\textwidth}
        \includegraphics[width=\textwidth]{images/plots/Plots/"scenario obs 00"/efficiency}
        \caption{Efficiency}
        \hfill
        \label{plot:efficiency plot scenario obs 00}
    \end{subfigure}
    ~
    \begin{subfigure}[b]{0.31\textwidth}
        \includegraphics[width=\textwidth]{images/plots/Plots/"scenario obs 00"/cost}
        \caption{Cost}
        \hfill
        \label{plot:cost plot scenario obs 00}
    \end{subfigure}
    ~
    \begin{subfigure}[b]{0.31\textwidth}
        \includegraphics[width=\textwidth]{images/plots/Plots/"scenario obs 00"/power}
        \caption{Power}
        \hfill
        \label{plot:power plot scenario obs 00}
    \end{subfigure}
    ~
    \begin{subfigure}[b]{0.31\textwidth}
        \includegraphics[width=\textwidth]{images/plots/Plots/"scenario obs 00"/turbines}
        \caption{Turbies}
        \hfill
        \label{plot:turbines plot scenario obs 00}
    \end{subfigure}
    \caption{Scenario obs00.xml averaged over 10 runs: (a) Fitness plot, (b) efficiency plot, (c) cost plot, (d) power plot, and (e) number of turbines.}
    \label{plot:scenario obs 00}
\end{figure}


\subsection{Results Scenario obs05.xml}


\noindent Figure \ref{plot:scenario obs 05} shows the results from scenario obs05.xml averaged over 10 runs. Fitness is shown in sub-figure \ref{plot:fitness plot scenario 05}. As can be seen, the results are similar to the results for the other 3 scenarios.\\


\noindent Since the result is similar to those observed before, they will not be discussed in details here, however one interesting observation has been made: In sub-figures \ref{plot:cost plot scenario obs 05} and \ref{plot:turbines plot scenario obs 05} it can be observed that every model end up in solutions with a different number of turbines. These results shows the complexity of the problem by showing how easy it is to end up in a local minima. Because, even thought the global minima is not known it is known that at least 3 of the models are stuck in a local minima.  \\


\begin{figure}[h!]
    \centering
      \begin{subfigure}[b]{0.31\textwidth}
        \includegraphics[width=\textwidth]{images/plots/Plots/"scenario obs 05"/best}
        \caption{Fitness}
        \hfill
        \label{plot:fitness plot scenario obs 05}
    \end{subfigure}
    ~
      \begin{subfigure}[b]{0.31\textwidth}
        \includegraphics[width=\textwidth]{images/plots/Plots/"scenario obs 05"/efficiency}
        \caption{Efficiency}
        \hfill
        \label{plot:efficiency plot scenario obs 05}
    \end{subfigure}
    ~
    \begin{subfigure}[b]{0.31\textwidth}
        \includegraphics[width=\textwidth]{images/plots/Plots/"scenario obs 05"/cost}
        \caption{Cost}
        \hfill
        \label{plot:cost plot scenario obs 05}
    \end{subfigure}
    ~
    \begin{subfigure}[b]{0.31\textwidth}
        \includegraphics[width=\textwidth]{images/plots/Plots/"scenario obs 05"/power}
        \caption{Power}
        \hfill
        \label{plot:power plot scenario obs 05}
    \end{subfigure}
    ~
    \begin{subfigure}[b]{0.31\textwidth}
        \includegraphics[width=\textwidth]{images/plots/Plots/"scenario obs 05"/turbines}
        \caption{Turbies}
        \hfill
        \label{plot:turbines plot scenario obs 05}
    \end{subfigure}
    \caption{Scenario obs05.xml averaged over 10 runs: (a) Fitness plot, (b) efficiency plot, (c) cost plot, (d) power plot, and (e) number of turbines.}
    \label{plot:scenario obs 05}
\end{figure}


\section{Discussion}\label{section:discussion}
\noindent In the sub-sections above, the results obtained for all 4 scenarios are presented. The major trends are listed below and will be discussed in this section.\\


\begin{enumerate}
    \item The Master/Slave model obtains the best fitness on all 4 scenarios.
    \item The Pool model obtains second best fitness on all 4 scenarios.
    \item The Island model obtains third best fitness on all 4 scenarios. \item The Cellular model obtains forth best fitness on all 4 scenarios. The model clearly stands out as unable to find an acceptable fitness in 200 generations. 
    \item The Master/Slave model, Pool model and Cellular model are not able to explore different number of turbines for more than about 10 generations.
    \item The Island model is the only model able to explore different solutions in the form of number of turbines. But, only for about 50 generations.
\end{enumerate}


\noindent The Master/Slave model obtained the best results on each scenario. This result was a little unexpected. According to \textcolor{red}{reference}, better results could be obtained with the Island model. One reason behind this could be that the parameters values and selection mechanism were selected based on results when tested on the Master/Slave model. Even though every model uses the same parameter values and selection mechanisms when possible, the optimal parameters for one model might not be the optimal parameters for another. One question that might come to mind is why was the parameters and selection mechanisms optimized for each model, and the answer to that questions is that it would not be possible to optimize these parameters for each model within the time frame of this thesis. In the future, this should be tested.\\


\noindent The Pool model performed second best on every scenario. Of all 3 population distributed genetic algorithms, the Pool model is the model that is most similar to the Master/Slave model and therefore the argument in above might also apply to the Pool model. Optimal parameters for the Master/Slave model is more likely to also be optimal parameters for the Pool model than the other two population distributed genetic algorithms. \\


\noindent Many reasons can be given to why the  Island model is outperformed by the Master/Slave model and the Pool model. In this thesis, the performance of the models are measured when each model runs for the same number of generations, and the same number of individuals are evaluated for each generation. In \textcolor{red}{reference}, it is only stated that better fitness can be obtained with the Island model, not that the two models are compared with the same resources. It is clear that if the Island model was run for 200 generations between each migration, with a population size of 100 individuals on each Island (same values as Master/Slave model) it would perform at least as good as the Master/Slave model, more likely better. The Island topology will also affect the results. If more Islands had been used, the population on each Island would be extremely small so that all individuals on each Island would soon be very similar. With a smaller topology however, the best individuals would need less generations to take over the entire population. The migration rate was kept at 2, meaning that less than 8\% of the individuals would be replaced at each migration. If a larger number was used the new individuals would spread and take over the Island faster, however, as the results show, with a tournament size of 20, the best individuals will take over an Island fast enough, so increasing the migration rate would only increase convergence, something that already is happening too fast. The parameter that the author believes that affect the performance of the Island model the most is the migration interval. The migration interval was kept at 20 so that 10 migrations would be performed. Meaning that individuals are able to travel around the circle 2.5 times. As can be seen from all the scenarios, after the first circle, the best individuals have taken over the entire population, and no other solutions are explored. The benefits of the Island model is therefore lost, and the model becomes a slower version of the Master/Slave model. The large tournament size used might be the reason why the best individuals are able to take over the population so fast. This problem could be solved by using a smaller tournament size on each Island or by using a longer migration interval so that the individuals are only able to travel around the circle once. The results show that the Island model is able to find the same solutions (number of turbines) as the Master/Slave model, but that it does not have enough time to optimize the turbine positions in the solutions. Because of this, the Island model does not realize that the current solution is will lead to better results later, and therefore it jumps out of the solution in order to explore a solution that seems more promising. A larger migration rate might have solved this problem.\\


\noindent The Cellular model is outperformed by all the other models. It is not able to get results that are close to those obtained by the others. On reason for this is that the Cellular model is by far the model that need most generations to find a good solution. For example, for the individual in the upper left corner, it would take a minimum of 14 generations to spread its genes to the individual occupying the lower right corner. What is even more scary is that on average it would take the same individual \textcolor{red}{how many?} generations to spread its genes to the lower right individual. Because of this slow converges, it is not surprising that the Cellular model is outperformed by the other models when they all run for 200 generations with 200 evaluation each generation. A solution to this problem would be to use a larger square for which parents for the middle individuals can be selected from. This would however make the model more similar to the Master/Slave model, and make it lose its unique features. As mentioned in chapter \textcolor{red}{reference if mentioned} for each generation every individual is replaced by a newly generated individuals. Something that explains why the fitness increases in the first 5-10 generations before it starts going down. This could be solved by implementing the Cellular model with the same replacement strategy as the Pool model; only replace individuals with new individuals with higher fitness. However, this was tested by the author on scenario 00.xml, and it actually lead to worse fitness than the current strategy. \textcolor{red}{Should these results be included? Appendix?}\\


\noindent As was mentioned in chapter \textcolor{red}{reference chapter, if it exists}, distributing the population is supposed to lead to more exploration because different parts of the population are supposed to explore different solutions. As shown in the results, the different population distributed genetic algorithms spent very little time in exploring different number of turbines. The Pool model and Cellular model spent less than 10 generations before solutions with a given number of turbines took over the population and optimization was reduced to moving turbines around in the farm. The Island model was the only model that were able to explore different numbers of turbines for a significant amount of time. As mentioned, the Island model used a little less than 50 generations before a given solution had taken over the entire population and the exploitation phase began. \\


\noindent The Pool model was not able to explore different solutions for a significant amount of time. As mentioned, the Pool model is asynchrounous by nature. The different workers were implemented as threads in Java. The Pool model simply starts all the threads and let them operate for 200 generations each without synchronization and without knowing about each other. The goal of this implementation is that different threads can be at different generations so that different parts of the Pool contains different solutions. The problem with implementing the workers as threads in Java is that the scheduling of the different threads are not under the control of the programmer. If the treads are interleaved (approximately) so that they are almost always at the same generation the model is basically reduced to the Master/Slave model. It is not known if this was the case, but it seems like a good explanation of why the threads are not able to explore different numbers of turbines and why its performance is so close to the performance of the Master/Slave model.\\


\noindent Since distributing the population implies that the models need more time to find the optimal solution, one might ask if it is fair to compare the different models with the same number of evaluations and same number of generations. The nature of the population distributed genetic algorithms is that they need more time to evolve. However, if these models are supposed to be used outside academia they can not be given unlimited resources. As mentioned above, it is clear that changing the deme size and migration interval to 100 and 200 respectively, it can not do worse than the Master/Slave model, but this would require more computational resource than available for this thesis.\\


\noindent Even though it is outside the scope of this thesis, it is interesting to discuss the fitness function. It is very hard to find the optimal fitness function. The fitness function is based on cost estimated, power estimates and wind predictions.Therefore, it might not be the case that those who came up with the fitness function is 100\% certain that it is correct. Because of this, they might be interesting in taking a look at the different solutions obtained by the different models even though the best fitness is obtained by the Master/Slave model. For example, in figure \ref{plot:scenario obs 00}, the Pool model is able to come up with a solution with fitness that is very similar than the fitness obtained by the Master/Slave model where the cost is largely reduced. Therefore, it might be interesting for the wind farm owners to take a look at different solution suggestions and compare them.\\


\noindent In section \textcolor{red}{reference correct chapter}. It is shown that the Master/Slave model performs best when the tournament size is extremely high; above 20\%. With a tournament size this high the best individuals will spread to the entire population fast. This implies that the final solution might not be the global minima, but that a local minima is found early in the evolution and that it is chosen before other solutions are explored. The author would expect a lower tournament size to obtain better fitness, because it would ensure that this would not happen. However, the results from \textcolor{red}{reference} showed that high tournament size worked best. This problem shows just how complex wind farm layout optimization is. So many different solutions exists, so it would be extremely unlikely that any algorithm would find the global minima. Therefore, optimizing a local minima seems like a better strategy. This reasoning might explain why the Master/Slave model outperforms the other models. Since the other models spend a little more time on settling in a local minima, a given number of turbines, the get fewer generations to optimize the turbine setting and therefore they are not able to keep up with the Master/Slave model.


